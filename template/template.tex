\documentclass[12pt]{exam}
\usepackage[a4paper, margin=1in]{geometry}
\usepackage{fontspec}
\usepackage{xeCJK}
\setCJKmainfont{SimSun}
\usepackage{header}
\usepackage{listings}

\title{第3讲\ 向量组}
\author{}
\date{\today}

\begin{document}
\maketitle

\section{Intro to Python}
\begin{questions}
\subimport{./}{questions.tex}
\begin{questionmeta}
  It's probably not necessary to get through all the parts of this problem if your students get the idea. The last subpart is probably the most instructive. 
  For question 2, feel free to remind students of the general ``order of operations'' of functions, in that they start inward and expand outward. Feel free to use any analogies that help students understand.
  If your students are still confused, it's advisable to carefully walk through the first few problems with them step by step. Encourage them to underline/annotate portions of the line to keep track of what parentheses go where!
\end{questionmeta}
\end{questions}

\end{document}
